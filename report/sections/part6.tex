\chapter{Adverting and Pricing Integration}

Reached this point the goal is to attract the most valuable users, that is allocate the budget during Advertising so that it is focused on attracting users that will maximize the total reward; this expectation on the reward is modeled by the pricing algorithms presented before. In this section we will present how we have modeled the intregation between Pricing and Advertising section, which are the algorithm proposed and an analysis of the final results.

\section{Proposed Integration}
The idea behind our integration is to have, for each Subcampaign, the possibility to manage separately the Advertising and the Pricing in order to reuse all the algorithms presented in the previous sections and, on top of it, build a Budget Allocator that is able to collect all the information and calculate the best allocation of the budget. The complete flow chart can be found at \ref{fig:part6scheme}, in the following we will discuss it step by step.

    \subsection{Budget Allocator}
    Every day, this component is the one that, collected information from the past, calculates the best budget allocation. Initially, when we have no historical information, it is built to output a balanced allocation among all the Subcampaigns. After the initialization phase, evey new day, it collects for every subcampaign $ j $ two values:
    
    \begin{enumerate}
        \item $ n_j (.) $ : Learned distribution of clicks over budget allocation for subcampaign $ j $. This function comes from the regression generated by the GPTS learner in the   \textbf{Advertising} section.
        \item $ v_j $ : Learned value, in terms of expected reward for subcampaign $ j $. This value comes from the evaluation of the collected reward during the last days in the \textbf{Pricing} section.
    \end{enumerate} 

    At this point we are able to define the Knapsack Optimization Problem as: 

    \begin{equ}[!ht]
        \begin{equation*}
            \max_{y_j,_t} \sum_{j=1}^{N} v_j n_j(y_j,_t)
        \end{equation*}
        \begin{equation*}
            \sum_{j=1}^{N} y_j,_t \leq \bar{y}_t
        \end{equation*}
        \captionsetup{labelformat=empty}
        \caption{The complete legend can be found at Figure \ref{fig:part6scheme}}
    \end{equ}


    \subsection{Advertising}
    For each subcampaign $ j $, given the budget allocation $ y_j,_t $, it is possible to collect from the environment the real number of clicks $ N_j(y_j,_t) $ generated by this allocation and update, as we have done in Chapter 2 and 3, the GPTS learner.
    At the end it is returned the distribution $ n_j(.) $ learned by the Gaussian Process.

    
    \begin{figure}[H]
        \centering
        \includegraphics[scale=0.8]{images/part6_schema.pdf}
        \caption{Flow Chart Diagram of part 6}
        \label{fig:part6scheme}
    \end{figure}


    \subsection{Pricing}
    For each subcampaign $ j $, given the real number of clicks $ N_j(y_j,_t) $ generated by the \textbf{Advertising}, it is possible to collect the real revenue $ r_j,_t $ fot the current day and update, as we have done in Chapter 4, the TS learner.