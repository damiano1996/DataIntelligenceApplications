\chapter{Assignments}

\begin{emptyenv}
    \label{assPart2}
    Part 2:\\
    \textit{Design a combinatorial bandit algorithm to optimize the budget allocation over the three subcampaigns to maximize the total number of clicks when, for simplicity, there is only one phase. Plot the cumulative regret.}
\end{emptyenv}

\begin{emptyenv}
    \label{assPart3}
    Part 3:\\
    \textit{Design a sliding-window combinatorial bandit algorithm for the case, instead, in which there are the three phases aforementioned. Plot the cumulative regret and compare it with the cumulative regret that a non-sliding-window algorithm would obtain.}
\end{emptyenv}

\begin{emptyenv}
    \label{assPart4}
    Part 4:\\
    \textit{Design a learning algorithm for pricing when the users that will buy the product are those that have clicked on the ads. Assume that the allocation of the budget over the three subcampaigns is fixed and there is only one phase (make this assumption also in the next steps). Plot the cumulative regret.}
\end{emptyenv}

\begin{emptyenv}
    \label{assPart5}
    Part 5:\\
    \textit{Design and run a context generation algorithm for the pricing when the budget allocated to each single subcampaign is fixed. At the end of every week, use the collected data to generate contexts and then use these contexts for the following week. Plot the cumulative regret as time increases. In the next steps, do not use the generated contexts, but use all the data together.}
\end{emptyenv}

\begin{emptyenv}
    \label{assPart6}
    Part 6:\\
    \textit{Design an optimization algorithm combining the allocation of budget and the pricing when the seller a priori knows that every subcampaign is associated with a different context and charges a different price for every context. Suggestion: the value per click to use in the knapsack-like problem depends on the pricing, that depends on the number of users of a specific class interested in buying the product. Notice that the two problems, namely, pricing and advertising, can be decomposed since each subcampaign targets a single class of users, thus allowing the computation of the value per click of a campaign only on the basis of the number of clicks generated by that subcampaign. Plot the cumulative regret when the algorithm learns both the conversion rate curves and the performance of the advertising subcampaigns.}
\end{emptyenv}

\begin{emptyenv}
    \label{assPart7}
    Part 7:\\
    \textit{Do the same of Step 6 under the constraint that the seller charges a unique price to all the classes of users. Suggestion: for every possible price, fix this price and repeat the algorithm used in Step 6. Plot the cumulative regret when the algorithm learns both the conversion rate curves and the performance of the advertising subcampaigns.}
\end{emptyenv}